\documentclass[UTF8]{ctexart}
\usepackage{amsmath}
\usepackage{geometry}
\usepackage{setspace}
\geometry{left=3.18cm,right=3.18cm,top=2.54cm,bottom=2.54cm}
\renewcommand{\baselinestretch}{1.5}
\pagestyle{plain}
\begin{document}
	\begin{center}
		\par \Large{\textbf{2020年第十二届大学生数学竞赛试题(非数学组)}}
		\par \normalsize\today
	\end{center}
	\textbf{一、填空题(共5小题,每小题6分,计30分)}
	\begin{enumerate}
		\item [(1)]极限$\lim\limits_{x\rightarrow 0} \dfrac{(x-\sin x)e^{-x^{2}}}{\sqrt{1-x^{3}}-1}$=\_\_\_\_\_\_\_\_\_\_.
		\item [(2)]设函数$f(x)=(x+1)^{n}e^{-x^{2}}$,则$f^{(n)}(-1)$=\_\_\_\_\_\_\_\_\_\_.
		\item [(3)]设$y=f(x)$是由方程$\arctan \dfrac{x}{y}=\ln \sqrt{x^{2}+y^{2}}-\dfrac{1}{2}\ln 2+\dfrac {\pi}{4}$确定的隐函数,且$f(1)=1$,则曲线$y=f(x)$在(1,1)处的切线方程为\_\_\_\_\_\_\_\_\_\_.
		\item [(4)]已知$\displaystyle\int_{0}^{+\infty}\dfrac{\sin x}{x}\,dx=\dfrac {\pi}{2}$,则$\displaystyle\int_{0}^{+\infty}dy\int_{0}^{+\infty}\dfrac{\sin x\sin (x+y)}{x(x+y)}\,dx$=\_\_\_\_\_\_\_\_\_.
		\item [(5)]设$f(x),g(x)$在$x=0$的一个邻域$U$内有定义,对$\forall x\in U,f(x)\neq g(x)$,且$\lim\limits_{x\rightarrow 0}f(x)=\lim\limits_{x\rightarrow 0}g(x)=a$,则$\lim\limits_{x\rightarrow 0} \dfrac{f(x)^{g(x)}-g(x)^{g(x)}}{f(x)-g(x)}$=\_\_\_\_\_\_\_\_\_.
	\end{enumerate}
	\textbf{二、(本小题满分10分)}\\
	设数列$\{a_{n}\}$满足$a_{1}=1,a_{n+1}=\dfrac{a_{n}}{(n+1)(1+a_{n})}(n\geq 1)$,求极限$\lim\limits_{n\rightarrow +\infty}n!a_{n}$.\\
	\textbf{三、(本小题满分10分)}\\
	设$f(x)$在[0,1]上连续,(0,1)上可导,且$f(0)=0,f(1)=1$.证明:\\
	(1)存在$x_{0}\in (0,1)$,使得$f(x_{0})=2-3x_{0}$;\\
	(2)存在$\xi ,\eta \in (0,1)$,使得$[1+f'(\xi)][1+f'(\eta)]=4$.\\
	\textbf{四、(本小题满分12分)}\\
	已知$z=xf(\dfrac{y}{x})+2y\varphi(\dfrac{x}{y})$,其中$f,\varphi$均为二次可微函数.\\
	(1)求$\dfrac{\partial z}{\partial x},\dfrac{\partial^{2}z}{\partial{x}\partial{y}}$;\\
	(2)若$f=\varphi$且$\left.\dfrac{\partial^{2}z}{\partial{x}\partial{y}}\right| _{x=a}=-by^{2}$,求$f(y)$.\\\\
	\textbf{五、(本小题满分12分)}\\
	计算$\displaystyle\oint_{\mathit{\Gamma}}\left|\sqrt{3}y-x\right|\,dx-5z\,dz$,其中曲线$\mathit{\Gamma}:\begin{cases}x^{2}+y^{2}+z^{2}=8\\x^{2}+y^{2}=2z\end{cases}$,从$z$轴正向向坐标原点看去取逆时针方向.\\
	\textbf{六、(本小题满分12分)}\\
	证明$f(n)=\sum\limits_{m=1}^{n}\displaystyle\int_{0}^{m}\cos\dfrac{2\pi n[x+1]}{m}\,dx$等于$n$的所有因子(包括1和$n$)之和,其中$[x+1]$表示不大于$x+1$的最大整数,并求$f(2021)$.\\
	\newpage
	\textbf{七、(本小题满分14分)}\\
	设$u_{n}=\displaystyle\int_{0}^{1}\dfrac{dt}{(1+t^{4})^{n}}\quad(n\geq 1)$.\\\\
	 (1)证明数列$\{u_{n}\}$收敛,并求$\lim\limits_{n\rightarrow +\infty}u_{n}$;\\
	(2)证明级数$\sum\limits_{n=1}^{+\infty}(-1)^{n}u_{n}$条件收敛;\\
	(3)证明当$p\geq 1$时$\sum\limits_{n=1}^{+\infty}\dfrac{u_{n}}{n^{p}}$收敛,并求级数$\sum\limits_{n=1}^{+\infty}\dfrac{u_{n}}{n}$的和.\\
\end{document}
